\documentclass[12pt]{article}  
\usepackage[spanish]{babel}       
\usepackage[letterpaper]{geometry} 
\usepackage[utf8]{inputenc}
\author{Jaziel David Flores Rodríguez.}       
\date{\today} 
\title{Fundamentos de Criptogrfía.} 


\begin{document}
\maketitle

\section{Introducctión.}

Tenemos que el desarrollo más sorprendente de la historia de la Criptografía fue en el año \textbf{1976}
de la mano de \textbf{Diffie} y \textbf{Hellman}, ellos publicaron \textbf{Nuevas Direcciones en Criptografía} por su traducción al español. Este artículo nos trajo al mundo el concepto revolucionario de \textbf{Llave Pública Criptográfica} y provió también un vuevo e ingenioso método para el intercambio de llaves, la seguridad de la cual esstá basada en la difucultad del logritmo discreto. Aunque los autores no teían un esquema páctico para la encriptación de una llave pública en ese momento. \\ 

Luego, \textbf{1978} Rivest, Shamir, y Adleman descubrieron el primer esquema de encriptación práctica y el esquema de firma, ahora denominado como \textbf{RSA}, el cual está basado en otro problema matemático, como la dificutad de factorizar números enteros.\\

Esta aplicación de problemas matemáticos difíciles a la criptografía revitalizó los esfuerzos para encontrar métodos más eficientes. En 1980 se  vieron los mayores avances en esta area y nadie pensó que el sistema \textbf{RSA} fuera inseguro. \\ 

Otra clase de poderosos y prácticos de esquemas de llaves públicas fueron encontrdos por \textbf{ElGama} en \textbf{1998}, este también está basado en el problema de logaritmo discreto. \\


Una de las contribuciones más significión proveídas por la criptografía de llave pública es la firma digital.\\ 

\pagebreak 

En \textbf{1991} el primer estándar de firmas digitales (ISO/IEC9796) fue adoptada, está basada en el esquema de llave pública RSA. En \textbf{1994} el gobierno de U.S.A. adoptó el Estándar de Firmas Digitales, un mecanismo basado en esquema de llave pública ElGamal.\\

La búsqueda de nuevos esquemas de llave pública, las mejoras existentes en los mecanismos criptográficos, y pruebas de seguridad continúan creciendo a un ritmo rápido. Varios estándares de infraestructuras se están poniendo en lugar en este momento y los productos de seguridad se están desarrollando para abordar las necesidades de una sociedad intensiva en información. \\ 

\section{Seguridad de la Información y Criptografía.}

El concepto de \textit{Información} se entenderá de manera cuantitativa, es decir que se refiere la naturaleza numérica de datos, métodos, investigaciones o resultados; tiene relación directa con cantidad, por lo tanto sus variables son siempre medibles, es decir no debe ser tremendamente  modificable por su interpretación.\\ 

La seguridad de la informción se manifiesta en muchas maneras, de acuerdo a la situación que se requiere, no importando quien o quienes están involucrados, en un grado u otro, todas las partes en una transacción deben tenere la confianza de que se han cumplido ciertos objetivos asociados con la seguridad de la información. A continuación se muestran algunos de estos objetivos.\\
	
\begin{table}[h]
\centering
	\begin{tabular}{ c | l  }
	\hline
\textbf{Objetivo} & \textbf{Función} \\ 
 & \\ \hline
\textbf{Privacidad o} & Mantener en secreto la información de todos  \\
\textbf{Confidencialidad} & aquellos que están autorizados a acceder. \\ \hline 
\textbf{Integridad de los Datos} & Asegurar que la información no haya sido altera- \\ 
 				 & da por perdonas o medios no autorizados. \\ \hline
\textbf{Autenticación de Entidad} & Corroborar la identidad o una identificación, como lo\\
\textbf{ o Identificación} & puede ser una persona o una tarjeta de crédito\\ \hline
\textbf{Mensaje de Autenticación} & Corroborar la fuente de la información; también \\
 & conocido como autenticación de origen de los datos. \\ \hline
\textbf{Firma} & Medio para vincular información a una entidad. \\ \hline
\textbf{Autorización} & Transmisión a otra entidad el permiso \\
		      & de la sanción oficial para hacer o ser algo.\\ \hline 
\textbf{Validación} & Es medio para proporcionar la puntualidad de la \\ 
		    & autorización para usar o manipular información o recursos. \\ \hline 
\textbf{Control de Acceso} & Restricción de accso a entidades privilegiadas. \\ \hline
\textbf{Certificación} & Respaldo de información por una entidad confiable. \\ \hline
\textbf{Timestamping o} & Registrar el tiempo de creación o existencia de \\
\textbf{Marca de Tiempo} & información \\ \hline
\textbf{witnessing} & Verifición de la cración o existencia de información \\ 
	            & por una entidad distinta del creador. \\ \hline
\textbf{Recibo} & Acuse de que la información ha sido recibida. \\ \hline
\textbf{Confirmación} & Reconocimiento de que se han proporcionado \\
		      & los servicios \\ \hline
\textbf{Ownership} & Ser el medio para proporcionar a una entidad el derecho legal \\ 
		  & de usar o transferir un recurso a otros. \\ \hline
\textbf{Anonimato} & Ocultar la identidad de una entidad involucrada \\ 
		   & en algún proceso. \\ \hline
\textbf{Non-repudiation} & Evitar la negación de compromisos \\ 
		         & o acciones anteriores. \\ \hline
\textbf{Revocación} & Retracción de la certificación o autorización. \\ \hline
	\end{tabular}
	\caption{\textit{Algunos Objetivos de la Seguridad de la Información}.}
\end{table}





\end{document}

